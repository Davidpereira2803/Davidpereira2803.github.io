%!TEX encoding = UTF8
%!TEX root =cv-llt.tex


\newcommand{\languagebar}[1]{%
  \begin{tikzpicture}
    \foreach \i in {1,...,5} {
      \ifnum\i>#1
        \fill[gray!30] (\i-1,0) rectangle (\i,0.4);
      \else
        \fill[black!70] (\i-1,0) rectangle (\i,0.4);
      \fi
    }
  \end{tikzpicture}%
}

\begin{rubric}{Skills}
\noentry{2014 -- 2015}
\entry*[Languages]
	\begin{itemize}
	    \item[\faFlag]\languagebar{3}\hspace{1em}German.
        \item[\faFlag]\languagebar{4}\hspace{1em}French.
        \item[\faFlag]\languagebar{4}\hspace{1em}English.
        \item[\faFlag]\languagebar{5}\hspace{1em}Luxembourgish.
        \item[\faFlag]\languagebar{5}\hspace{1em}Portuguese (Native).
	\end{itemize}
\entry*[Programming\hfill]
	Java, Python, JavaScript, HTML, CSS, C++, C, \LaTeX, \ldots
\entry*[Web Dev]
	\textsc{Html, css}, JavaScript, FastAPI, React.
\entry*[Robotics/Electronics]
    ROS, Linux, Raspberry Pi, DJI Tello Drone
\entry*[Data Base Technologies]
    FireBase, SQLite
\entry*[Additional Interests]
	Academic research, Football, Sport.
\end{rubric}
